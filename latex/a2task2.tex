\subsection{Tuning of ESKF for simulated data}

We initialized the tuning by looking up typical values for GNASS and IMUs. For the IMU and gyroscope input models we used the datasheet for STIM300 to select the continuous time noise parameters which we then attempted to modify a bit to improve NEES. We tuned the standard deviation for GNASS using \texttt{eskf.NISGNSS} and attempted to get it within the $95\%$ CI. We increased the standard deviation for the altitude component in the GNSS measurement noise since GNSS is usually best at estimating XY position. For the bias models we selected a reciprocal time constant $p_{ab} = p_{\omega b} = 10^{-8}$ to model the biases as close to random walk. The bias noises were then hand tuned to get NEES for bias within the $95\%$ CI. We also looked at the estimation errors and made sure they remained reasonably low. We had extra focus on the attitude error because it propogates into the acceleration in the intertial frame (which causes a lot of errors when integrated twice for position). The error state covariance was initialized to a quite low value to a long period of large errors in the beginning until the filter has converged. Initialized is important for ESKF as it can  converge very slow if initialized incorrectly. 
\begin{subequations}
\begin{equation}
q_a = (1.167 \cdot 10^{-3})^2, \\
\end{equation}
\begin{equation}
q_{ab} = (1.5 \cdot 10^{-3})^2, \\
\end{equation}
\begin{equation}
p_{ab} = 0, \\
\end{equation}
\begin{equation}
q_\omega = (2.5 \cdot 10^{-2})^\circ, \\
\end{equation}
\begin{equation}
q_{\omega b} = (2 \cdot 10^{-5})^2, \\
\end{equation}
\begin{equation}
p_{\omega b} = 0, \\
\end{equation}
\end{subequations}


