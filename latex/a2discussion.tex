\subsection{Discussion}
This sections contains some thoughts on ESKF from tuning the filter for the simulated and the real dataset.



\subsubsection{NIS}
Interesting to note about the NIS, are the seemingly periodic spikes that leave the confidence interval. By superimposing the NIS value over the 2D-projected AUV trajectory, like shown in figure \ref{fig:nis_colored_track} for the real dataset, it is evident that the spikes appear when the AUV is turning. This most likely stems from the fact that turns are non-linear motion, and the ESKF uses a linearization model. We also note the large spike in NIS early in figure \ref{fig:nis_basic}. This spike appears at $t=206s$, right as the AUV is catapulted into the air, as such it is natural to expect a large innovation value. When tuning the noise parameters we therefore have to balance the NIS between the different behaviours and it is difficult to get a "one-fits-all" solution, if not impossible. 

\subsubsection{Attitude estimation (task 3B)}
The attitude estimation for roll and pitch is very precise which makes sense since they are observable using the IMU due to the constant gravity vector. Yaw is however not possible to observe without additional requirements on the measurements. Without any sensor for measuring global heading (i.e compass) we need to get the heading from the velocity, which we can estimate from the GNSS measurements \textit{if the plane is moving.} If the velocity vector is zero, we have no information about the direction and can not estimate the heading. (i.e lack of suffecient exciation) 

\subsubsection{IMU Mounting errors (task 3C and 4B)}
Mounting errors of the IMU will affect how we define the body frame. When using other sensors in the body frame it is important that we know how they all relate to the body frame so get accurate measurement models. If the IMU is mounted incorrectly in our case, the specified leverarm for the GNSS receiver will not be correct and it will cause loss of accuracy. When using $S_a = S_g = I$ we saw a loss in accuracy and worse consistency. But not as much as we expected. We only rely on the attitude of the body frame when transforming the IMU measurement during the prediction, and when updating the estimate using GNSS \textit{due to the leverarm.}. The prediction step would be almost completely unaffected by mounting errors of the IMU as long as the filter is given enough time to converge. As long as the IMU knowns its own orientation it is able to transform the measurements into world frame. The GNSS receiver also depends on the attitude due to to leverarm of the antenna. The relatively large uncertianty does however reduce the impact small mounting errors of the IMU might have. By this we conclude that the IMU should be mounted correctly (or compenstate for any errors in software), but small errors will not completely break the filter. 

Misalignment of the coordinate axes of the IMU from manufacturing is however a bit worse since we then have correlation between the different axis. Acceleration in one axis will"contaminate" the measurements for the misaligned axis and therefore violate the assumption that the IMU sensor noise is white (we no longer get zero mean due to the added contamination from the misaligned axis). It is therefore important to correct for misaligned errors as much as possible. When setting the correction matrix to the identity matrix we saw increased error in the filter and reduction in consistency.  




