\subsection{Choice of JCBB CI Bounds with Implication on Robustness}
The JCBB algorithm operates with two confidence intervals, one for joint and one for individual compatibility. The Mahalonobis distances of both joint and individual compatibility are gated with these confidence intervals to determin whether to move on with this hypothesis.\cite{jcbb} In the Victoria Park dataset, we found that the choice of large confidence intervals ($\alpha_1 = 10^{-5}$ and $\alpha_2 = 10^{-3}$ for joint and individual respectively) gave perfectly satisfactory association speed on the dataset, however when EKF-SLAM was rerun on the generated map with a slight offset in orientation of $6^\circ$ (such a situation could appear naturally from disturbances such temporary loss of sensor data), the JCBB algorithm wound up in an almost complete stall after a few hundred timesteps, likely because all hypothesis were regarded as more equally likely due to the offset, making the algorithm unable to reduce the search tree and single out the best association in reasonable time. Choosing tighter confidence intervals on the other hand, with $\alpha = 0.1$ for both joint and individual, many more associations were gated out and hence only the most promising hypotheses were considered, ultimately leading to reasonable execution times. In the end, this made the rerun on the previously built map perfectly possible with no significant slow-down. This result shows how the choice of these confidence intervals is detrimental to the robustness of EKF-SLAM and crucial for online SLAM as a small disturbance can cause the JCBB algorithm to fail to find associations within reasonable time.

