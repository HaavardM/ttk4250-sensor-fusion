\subsection{Tuning of EKSF-SLAM for simulated dataset} \label{a3-sim-tuning}
We started by tuning the odometry noise $Q$ and measurement noise $R$ by using NEES and NIS to get the initial tuning. NIS and NEES was normalized to get a value between $0$ and $1$ due to the increasing degrees of freedom when adding new landmarks. We used a diagonal $Q$ and tuned the translation and heading independently to improve NEES. The measurement noise $R$ was tuned by attempting to maximize NIS while keeping the entries in $R$ as low as possible. We also created a plot for the position error over time to keep track of the estimation error. We realized that there were a tradeoff between a having a consistent filter (acceptable NIS and NEES) and the estimation error. Trying to improve the NEES resulted in increased error and trying to reduce the estimation error gave us worse NEES and NIS. We decided to prioritize the estimation error when tuning, but also kept an eye on the NEES and NIS to make sure we remained close to the confidene intervals. With our finished tuning we achieved a $82\%$ within NIS CI and $68\%$ within NEES CI, while also keeping the estimaion error reasonably low (less than $0.5m$). The JCBB alphas, $\alpha_1$ and $\alpha_2$, was choosen not to low to avoid making too many wrong associations. Otherwise it was tuned by balancing it with the other parameters while looking at NEES, NIS and estimation error. 

\begin{tcolorbox}[ams align, title={ESKF-SLAM tuning for simulated dataset}]
    Q &= \begin{bmatrix}(5.32*10^{-2})^2 & 0 & 0 \\0 & (5.32*10^{-2})^2 & 0 \\0 & 0 & (1.4*10^{-2})^{2} \end{bmatrix} & R &= \begin{bmatrix}(3.2*10^{-2})^2 & 0 \\0 & (3.2*10^{-2})^2\end{bmatrix} \\
    \alpha_{1} &= 10^{-6} & \alpha_2 &= 10^{-3}
\end{tcolorbox}


\begin{figure}
    \centering
    \begin{subfigure}{0.45\textwidth}
        \includegraphics[width=\textwidth]{plots/a3-sim-nis}
        \caption{NIS}
        \label{fig:a3-sim-nis}
    \end{subfigure}
~
    \begin{subfigure}{0.45\textwidth}
        \includegraphics[width=\textwidth]{plots/a3-sim-nees}
        \caption{NEES}
        \label{fig:results_bg}
    \end{subfigure}
    \begin{subfigure}{0.45\textwidth}
        \includegraphics[width=\textwidth]{plots/a3-sim-error}
        \caption{Error}
        \label{fig:a3-sim-error}
    \end{subfigure}
~
    \begin{subfigure}{0.45\textwidth}
        \includegraphics[width=\textwidth]{plots/a3-sim-results}
        \caption{Track}
        \label{fig:a3-sim-result}
    \end{subfigure}
    \caption{EKF-SLAM results for simulated dataset}
\end{figure}