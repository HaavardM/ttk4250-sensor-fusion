\subsection{IMM-PDAF on the real data set "Joyroide"}
Tuning of IMM-PDAF for the real dataset joyride was a bit more tricky. In the simulated dataset there were more distinct sections of either turning or constant velocity. We could also generate independent datasets for tuning each model. On the real dataset we have to model everything at once and it is challenging to get a robust track at all. We initialized the filter using the parameters from task 2 which resulted in loss of track after only a few turns. We introduced a third model $CV_{high}$ (high model noise $qCV_{high}$) which caputures the dynamics not well explained by our CV or CT models. (e.x high linear acceleration). Once we found some values that gave us a complete track we could try an iterative approach to improve NEES and the estimation error. We set an $95\%$ CI for the NEES and tried small variations in all parameters. The clutter rate $\lambda$ and detection probability $P_D$ had to be reduced quite a bit, while the measurement noise $r$ had to be increased to get a robust track. The process noise $qCV$ and $qCT$ also had to be increased quite a bit compared to task 2. When tuning the process noise $qCV$ and $qCT$ we made sure that there was a clear distinction in the mode probabilities between the CV and CT models. The colored track plot from task 2 was also used to visualize when the different modes were prefered by the filter. The process noise $qCV_{high}$ was set a lot higher to make sure it was left out unless none of the other models could explain a set of measurements.
The Markov Matrix $PI$ was tuned to prefer the CV and CT models. 
\begin{figure}
    \centering
    \hspace*{-2cm}\begin{adjustbox}{minipage=\linewidth, scale=1}
        \begin{subfigure}{.5\textwidth}
            \includegraphics[width=\linewidth]{plots/a1/task3/a1_task3_error.eps} 
            \caption{Tracking error}
            \label{fig:task3_tracking_error}
        \end{subfigure}
        \begin{subfigure}{.5\textwidth}
            \includegraphics[width=\linewidth]{plots/a1/task3/a1_task3_NEES.eps} 
            \caption{NEES}
            \label{fig:task3_NEES}
        \end{subfigure}
    \end{adjustbox}
        \caption{Tracking performance when using the real dataset "Joyride"}
\end{figure}