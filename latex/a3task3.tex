\subsection{Victoria Park Dataset}
The Victoria Park dataset is a classic dataset for 2D SLAM that has been analyzed numerous times in the literature. The dataset is generated by a car driving in Victoria Park, Sydney detecting trees with a front-mounted laser scanner and recording its odometry using encoders and steering sensors. When tuning for this dataset we initialized the filter with the standard deviations found using the simulated dataset. We used the same normalized NIS as we did in \cref{a3-sim-tuning} and tried to get as consistent results as possible by targeting the amount of NIS within our $95\% \chi^2$ CI. The provided GPS measurements was missing large gaps of data, and due to the low precision in raw GPS measurements we argue it is not good enough to be used as ground truth. NEES is therefore considered unavailable. As shown in \cref{fig:results_bg}, we overlayed a picture from google maps over our track and aligned the landmarks as best as we could. By considering the visual overlay and NIS we tuned our filter to get as precise results as possible while still getting a consistent filter. Our finished tuning resulted in $37\%$ within the NIS CI. Knowing however that the EKF has an issue with consistency for SLAM, we decided to focus on providing robust tracking with little deviation from the GNSS measurements instead of chasing a perfectly consistent filter. As can be seen in figures \ref{fig:a3-real-results_bg} and \ref{fig:a3-real-gnss-diff}, the tracking is satisfactory and deviates by at most 10 meters from the GNSS measurements. It should be noted that GNSS measurements cannot in general be regarded as a true ground truth. It is also aligned by hand in our case, further contributing to the error.
